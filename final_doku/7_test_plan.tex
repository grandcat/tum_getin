\section{Test Plan}\label{sec:test_plan}

\todo{struktur!}
Test coverage, Test methods, Test responsibilities

...Das Ganze auf den Funktionen aus den letzten Kapiteln basieren lassen....\\


Due to the strict separation of the components \be, \ph and \ter testing could partly be initiated in early stages of the project.

Particularly the \be component was required to quickly reach a robust state because both other components heavily rely on its functioning.
In order to guarantee this robustness, development followed a test-driven approach from the beginning.
Functionality required in the \be was first defined in a specifications document.
Subsequently test cases based on these specifications were written using \vows\footnote{Vows: Asynchronous behaviour driven development for Node. http://www.vowsjs.org}.
Only then did we begin to actually implement the defined functionality.

...\todo{...}


Test scenarios: high level\\
- user has TumOnline account, is student\\
- user has TumOnline account, not student anymore\\
- user has no TumOnline account\\

low level:\\
- testing all possible mistakes: requests with wrong / missing arguments\\
- sending broken / incomplete packets\\
- interrupting connections\\
- possible failures of involved systems. E.g. TumOnline token max reached, AD not responding, general connection problems, ...
...\\


...

%-------- from Wikipedia: -----------------
\iffalse
IEEE 829-2008, also known as the 829 Standard for Software Test Documentation, is an IEEE standard that specifies the form of a set of documents for use in defined stages of software testing, each stage potentially producing its own separate type of document.[1] These stages are:

Test plan identifier
Introduction
Test items
Features to be tested
Features not to be tested
Approach
Item pass/fail criteria
Suspension criteria and resumption requirements
Test deliverables
Testing tasks
Environmental needs
Responsibilities
Staffing and training needs
Schedule
Risks and contingencies
Approvals
\fi
