\section{Test Plan}\label{sec:test_plan}

%\todo{struktur!}
%Test coverage, Test methods, Test responsibilities
%...Das Ganze auf den Funktionen aus den letzten Kapiteln basieren lassen....\\


Due to the strict separation of the components \be, \ph and \ter testing could partly be initiated in early stages of the project.

Particularly the \be component was required to quickly reach a robust state because both other components heavily rely on its functioning.
In order to guarantee this robustness, development followed a test-driven approach from the beginning.
Functionality required in the \be was first defined in a specifications document.
Subsequently test cases based on these specifications were written using \vows\footnote{Vows: Asynchronous behaviour driven development for Node. http://www.vowsjs.org}.
Only then did we begin to actually implement the defined functionality.

\medskip

\noindent
For testing all interfaces to and from the backend you may also refer to the specifications document ($spec.txt$) which lists all possible server messages, including all possible responses to potential expected and unexpected errors.

\bigskip

\noindent
High-level testing of the entire \app solution was done by using three different scenarios:
\begin{itemize}
\item A user has a valid TUMOnline account and is student.
\item A user has a valid TUMOnline account but is not student any longer.
\item A user has no valid TUMOnline account.
\end{itemize}

\noindent
On a low level, we tried to cover all possible causes of errors or unexpected behaviour.
Testing included:
\begin{itemize}
\item Trying requests with wrong or missing arguments.
\item Sending broken or incomplete packets.
\item Interrupting connections.
\item Testing with expected error messages of involved systems.
E.g.~TUMOnline may send an error message if a user has reached his token limit (10).
\item Testing possible unexpected failures of involved systems like connection problems to TUMOnline or the Active Directory.
\end{itemize}


\app's global test plan is based upon the high-level scenarios defined above and the list of functionality presented in chapter \ref{sec:functions}. It looks as follows:
\bigskip

\noindent
\begin{tabularx}{\textwidth}{ X X X c } 
Test case & Expected result & Result & OK? \\ \hline\hline

Ph $\rightarrow$ BE: Register with valid TUM ID & BE returns token, pseudo ID and salt & $\rightarrow$ BE returns token, pseudo ID and salt & \checkmark \\ 
 &  & $\rightarrow$ BE returns error & $\times$ \\ 
 &  & $\rightarrow$ No answer & $\times$ \\ 
 &  & $\rightarrow$ BE or Ph crash & $\times$ \\ \hline

Ph $\rightarrow$ BE: Ask if token is activated & BE returns $true/false$ & $\rightarrow$ BE returns $true/false$ & \checkmark \\ 
 &  & $\rightarrow$ BE returns error & $\times$ \\ 
 &  & $\rightarrow$ No answer & $\times$ \\ 
 &  & $\rightarrow$ BE or Ph crash & $\times$ \\ \hline

Ph $\rightarrow$ BE: Send $K_{pub}$ & BE returns OK & $\rightarrow$ BE returns OK & \checkmark \\ 
 &  & $\rightarrow$ BE returns error & $\times$ \\ 
 &  & $\rightarrow$ No answer & $\times$ \\ 
 &  & $\rightarrow$ BE or Ph crash & $\times$ \\ \hline

Ph $\rightarrow$ BE: Renew pseudo ID and salt  & BE returns new pseudo ID and salt & $\rightarrow$ BE returns new pseudo ID and salt & \checkmark \\ 
 &  & $\rightarrow$ BE returns error & $\times$ \\ 
 &  & $\rightarrow$ No answer & $\times$ \\ 
 &  & $\rightarrow$ BE or Ph crash & $\times$ \\ \hline

Ph $\rightarrow$ BE: Remove user account & BE returns OK & $\rightarrow$ BE returns OK & \checkmark \\ 
 &  & $\rightarrow$ BE returns error & $\times$ \\ 
 &  & $\rightarrow$ No answer & $\times$ \\ 
 &  & $\rightarrow$ BE or Ph crash & $\times$ \\ \hline

\end{tabularx}

\bigskip
\noindent
\begin{tabularx}{\textwidth}{ X X X c } 
Test case & Expected result & Result & OK? \\ \hline\hline

Ter $\rightarrow$ BE: Ask for $K_{pub}$ of a pseudo ID & BE returns $K_{pub}$, H(token) and salt & BE returns $K_{pub}$, H(token) and salt & \checkmark \\ 
 &  & $\rightarrow$ BE returns error & $\times$ \\ 
 &  & $\rightarrow$ No answer & $\times$ \\ 
 &  & $\rightarrow$ BE or Ter crash & $\times$ \\ \hline


\end{tabularx}

\bigskip
\noindent
\begin{tabularx}{\textwidth}{ X X X c } 
Test case & Expected result & Result & OK? \\ \hline\hline

Enter no ID at step 1 & Please enter ID message & as we expected & \checkmark \\ 
Enter non existent ID at step 1 & Wrong ID message & as we expected & \checkmark \\ 
Enter real ID & Backend connection and forwarding to step 2 & as we expected & \checkmark \\ 
Swipe through pages without entering an ID & Display status correctly at all times & as we expected & \checkmark \\ 
Open/resume app & crosscheck user status with backend and display the appropriate page & as we expected& \checkmark \\ 
Resume app after token activation & update step 2 and forward the user to register complete & as we expected & \checkmark \\ 
Delete token by hand, resume to app & pages need to be updated properly, user gets forwarded to step 1 & as we expected & \checkmark \\ 

Select settings menu at any time & open settings page & rarely sporadically, the app crashes & $\times$ \\ 
\end{tabularx}




%-------- from Wikipedia: -----------------
\iffalse
IEEE 829-2008, also known as the 829 Standard for Software Test Documentation, is an IEEE standard that specifies the form of a set of documents for use in defined stages of software testing, each stage potentially producing its own separate type of document.[1] These stages are:

Test plan identifier
Introduction
Test items
Features to be tested
Features not to be tested
Approach
Item pass/fail criteria
Suspension criteria and resumption requirements
Test deliverables
Testing tasks
Environmental needs
Responsibilities
Staffing and training needs
Schedule
Risks and contingencies
Approvals
\fi
