\bigskip
\bigskip

%\newpage

\section{Introduction}

This paper presents the final version of \app, a solution to control access to university buildings via a mobile device communicating with a receiver at the door.
This communication works via near field communication (NFC).
Main contribution of the presented project is an Android application embedded in a holistic system which provides simple door access capabilities.
%Given the limited scope of the project, we start with a prototypical setup while striving for the system to eventually be employed by the TU München (TUM).

%Apart from this application we plan a prototypical implementation of a NFC reader using 



\subsubsection*{Problem description}

%\todo{ganzes Kap kürzen!}

Full-time students at TU München often face the problem that courses or seminars take place on a Saturday or a Sunday.
Since buildings at the university campus are generally closed at weekends, students have to ring for the security personnel to open the door.
But due to the number of students entering and leaving university buildings, door wardens hardly ever check the legitimacy of any entrant's business, i.e.~if they really are students.
This situation is unsatisfactory for all participants:

\begin{itemize}
\item Students needlessly have to wait for someone to open the door.
\item Door wardens get distracted from more important work.
\item Door wardens and university authorities cannot guarantee that entering people are students (or authorised persons in general).
\end{itemize}

\subsubsection*{Solution}

In order to overcome afore-mentioned problems, this paper presents a solution which allows students to authenticate at a door system and are granted or denied access automatically. This approach helps to...

\begin{itemize}
\item reduce possible waiting times for students,
\item unburden door wardens,
\item and guarantee that only students and other authorised persons are granted access.
\end{itemize}

A few areas in TUM buildings already use an access mechanism via the student card.
But for security reasons this student card solution is not ideal.
Because the card only identifies via an imprinted unique ID, reading and duplicating the card is possible.
For this reason we developed a smartphone solution which implements state of the art security mechanisms.
This solution has the potential to not only be used for main door entrances to the university but also to control access to chairs or other restricted areas.
It is an overall solution, relying solely on the user's smartphone.
 
 \subsubsection*{Structure of this paper}

This paper continues with a brief introduction of the project team, followed by the project plan (chapter \ref{sec:plan}).
Several problems we faced during the project are mentioned in chapter \ref{sec:project_flow}.
After an overview of \app's architecture in chapter \ref{sec:arch}, section \ref{sec:impl} presents more technical insights into our solution, including details of the Android application and the NFC communication between smartphone and NFC terminal.
Subsequently, all implemented functionality is listed in chapter \ref{sec:functions}, followed by a detailed test plan.
After a conclusion we also appended a tutorial for everybody who wants to try \app.

%The following chapter motivates the need for such a door access solution and sketches the project idea. Chapter \ref{sec:arch} then details the proposed technical architecture. Due to a several possible approaches, chapter \ref{sec:alt} justifies our design decisions. Finally, chapter \ref{sec:team} introduces all team members and chapter \ref{sec:plan} outlines a project plan.\todo{Anpassen} 
