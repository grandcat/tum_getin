\section{Project Flow}\label{sec:project_flow}

Concerning the implementation of all planned functionality of \app we faced few obstacles.
Most issues of the initial project plan could be finalized smoothly.

In several details the functionality could even be improved by \textbf{additional features}.
The following list explains all important changes and new features:
\begin{itemize}
\item The smartphone app can now ask the backend if the user has already activated his token in TUMOnline.
This ensures that the smartphone does not constantly try to proceed with the next step (upload public key) but of course always fails because only activated user accounts are allowed to perform this task.
\item A user can now ask the backend to generate a new pseudo ID for him.
This effectively increases the user's privacy, as NFC terminals are no longer able to track his actions.
\item Users can now delete their accounts from the backend's database, which in general should be a useful feature.
\item In contrast to the original project plan we now hash the user's token (SHA256 + salt) whenever communicating with the NFC terminal.
As a consequence the attack surface for learning the token is effectively reduced because this token is never sent via NFC nor via Ethernet to the terminal.
The terminal never learns the user's token, only a salted hash of it.
%\item ...?\todo{mehr?}
\end{itemize}

\bigskip


But as in most projects we also faced a few setbacks and \textbf{problems}.
Our original ambition was to see the \app solution employed in a real environment at TUM.
To accomplish this goal we had several meetings with Mr.~Bernhofer from the TUM IT Management and later with Mr.~Recksiegel from the TUM physics department.
The latter already runs several door terminals which allow registered users to enter rooms by holding their student cards to a NFC reader.
As we did not and still do not want to promote our solution as competition to the solution already employed in some rooms of the physics department, we searched for a way to integrate \app into the existing system of the TUM physicists.
For several reasons this turned out to be more difficult than expected.
\begin{itemize}
\item TUM student cards communicate via proprietary NFC standards from the company LEGIC, which are not compatible with the NFC ISO standards that Android smartphones understand.
\item The NFC terminal built into the TUM physics solution is not able to communicate via NFC standards other than a restricted set of LEGIC commands.
\item Since the NFC terminal used by the TUM physics department is a proprietary product and ships with a rather small housing, we did not manage to easily mount a second NFC antenna into its housing.
\item Integrating two different NFC units into the door terminal entailed several physical problems.
For two NFC units to function independently they would need to be mounted with at least about five centimetres distance to each other.
This led to concerns that a terminal with two distinct NFC terminals might not be intuitive for users.\\
Another idea would be to mount the two NFC units into one housing but keep only one of them running at a time.
If the NFC unit which is running by default (the ISO standards transceiver) detects a TUM student card, it sends a command to switch to the LEGIC reader.
After the LEGIC transceiver has handled NFC communication with this student card it is switched off again and the default NFC reader is listening again.
If the default NFC units detects a smartphone, it can handle the communication itself.
%\item Diese Liste muss in Details noch überarbeitet werden! Bis hier aber schon ganz gut =)\todo{!!!}
%\item Ist not etwas wirr und unstrukturiert...
%\item ...hab ich was vergessen? Vll, ein zukünftiges Ziel wäre die parallele Integration mit abwechselndem Betrieb einzuführen\todo{???}
\end{itemize}


% -------------- Kommentare und alter Mist ----------------------- %
\iffalse

:\todo{TODO}
hatten wir ja ein paar: Zusammenspiel mit anderen Lesegeräten (Legic), ...
-> dadurch: unwahrscheinlicher, dass die Lösung in der Praxis eingesetzt werden wird.

any problems? Differences to the planned functionality?

Differences to the planned functionality:
gibt's bei uns eher nicht, oder?

Stefan: Ne, eigentlich sogar noch mehr Funktionalität und erhöhte Sicherheit/Anonymität als spezifiziert.
\fi