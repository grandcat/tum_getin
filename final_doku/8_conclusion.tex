\section{Conclusion}\label{sec:conclusion}

The presented area access solution is an easy and comfortable way for users to authenticate themselves via smartphone NFC communication.
Even though the \app solution currently only uses the immatriculation status of the users, it can easily be adapted for more advanced rules.
The current focus on secure authentication could then be extended by useful authorization mechanisms.
Linking further user information in a database would allow scenarios like ``chair access restricted to chair members only''.

Since \app is built modularly, it can easily be integrated into the existing TUM-Campus-App to provide an overall student solution.

When proceeding into a production phase the backend of course will need to be hosted on a TUM server.
Additionally, the backend's current X509 certificate would need to be replaced by one that is signed by the TU München.
Since we employ \textit{key pinning} for additional security, this will also entail changes in the Android app and the NFC terminal.

A productive version should also contain the possibility to interoperate with a LEGIC transceiver as explained in section \ref{sec:project_flow} in order to provide smartphone and student card NFC communication in one housing.
This would require an additional small script on the NFC terminal which manages and switches the two NFC readers.
Backend and Android app are not affected by this change.

Altogether, after these little additional steps \app is a mature solution and ready to be employed in practice.


\iffalse
Outlook?

- Die Lösung im Zusammenspiel mit dem Legic-Zeugs beschreiben. Das mit zwischen den zwei Readern hin- und herschalten.
Wobei Teile dieser Problembeschreibung schon ins Kapitel ``project flow'' müssen.

- Was man in der Praxis machen müsste: halbfertige Registrierungen (Token nicht aktiviert) nach einiger Zeit wieder aus dem Backend löschen. Sonst könnte jemand das Backend total zumüllen. DoS-mäßig...

- (eventuell) Man müsste IP-Adressen von requests im backend loggen. Sonst könnte man mittels Backend als Proxy TUMOnline spammen.

\fi