\section{Conclusion}\label{sec:conclusion}

The represented area access solution is an easy and comfortable way for users to authenticate themself. Even though the solution in this case only uses
the immatriculation status of the users, it can easily be adapted for more advanced rules if the information is linked to the users in a database (e.g. chair access restricted to chair members).
The app is built modular, so it can be easily integrated into the existing TUM-Campus-App to provide an overall student solution. To get the system out of the prototype stage, the backend needs to be hosted at a TUM-Server with  a valid not self signed certificate.


Outlook?

- Can be adapted for more advanced rules. Do more fine-grained access control. E.g. doors at chair X only open if someone from chair X is allowed to.

- Die Lösung im Zusammenspiel mit dem Legic-Zeugs beschreiben. Das mit zwischen den zwei Readern hin- und herschalten.
Wobei Teile dieser Problembeschreibung schon ins Kapitel ``project flow'' müssen.

- Könnte man in die TUM-Campus-App integrieren...

- Was man in der Praxis machen müsste: halbfertige Registrierungen (Token nicht aktiviert) nach einiger Zeit wieder aus dem Backend löschen. Sonst könnte jemand das Backend total zumüllen. DoS-mäßig...

- Natürlich müsste man das Backend irgendwo auf einem TUM-Server hosten und neue Zertifikate verwenden (von der TUM ausgestellt und signiert).

- (eventuell) Man müsste IP-Adressen von requests im backend loggen. Sonst könnte man mittels Backend als Proxy TUMOnline spammen.