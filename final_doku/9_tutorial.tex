\section{Tutorial}\label{sec:tutorial}

Before actually using the \app solution on a smartphone, both the \be and a \ter have to be running.
Hence this chapter starts with installation instructions for \be and \ter.

General note: a command line starting with \# indicates a root shell, a \$ symbol indicates one without super-user privileges.

\subsection{Install and Launch Backend Server}

The server was implemented based on the JavaScript (JS) technologies \node and \textit{ME(A)N.js} (M = \mongo, E = $express.js$ [web server], A = $angular.js$ [currently not used], N = \node).
Hence it requires a running \mongo database and \node to be installed.
This chapter serves as a recommendation for starting the server.
Experienced users may also install some components locally (without the $-g$ option) or omit some parts.\medskip\\
1.) Install \node.\smallskip\\
2.) Install $npm$, the Node Package Manager.\smallskip\\
3.) Download \mongo.\smallskip\\
4.) Start \mongo via
\begin{lstlisting}
   $ <path-to-mongodb>/bin/mongod
\end{lstlisting}
5.) Install $bower$, a package manager.
\begin{lstlisting}
   # npm install -g bower
\end{lstlisting}
6.) Install $grunt$, a JS task runner for automation.
\begin{lstlisting}
   # npm install -g grunt-cli
\end{lstlisting}
7.) Install $vows$, a JS framework for writing and executing test suites. This is only necessary if tests need to be run.
\begin{lstlisting}
   # npm install -g vows
\end{lstlisting}
8.) Install all \node dependencies of the \be server. $npm$ automatically loads all dependencies defined in $package.json$ and stores them locally.
\begin{lstlisting}
   $ npm install
\end{lstlisting}
9.) Finally, \textbf{run the server} via the $grunt$ client (optionally with the environment variable \textit{NODE\_ENV=test}).
\begin{lstlisting}
   $ grunt
\end{lstlisting}\bigskip
(10.) Optionally, run the $vows$ test suite (change to $/test$ directory first; \textit{NODE\_ENV=test} is required for some unit tests).
\begin{lstlisting}
   $ vows interfaces-test.js --spec
\end{lstlisting}

\subsection{Install and Launch NFC Terminal}

...

\subsection{Use the GetInTUM Android Application}

...



\iffalse
\bigskip
% -------------------------------------------------------------------
%[language={[x86masm]Assembler}, escapeinside={(*§}{§*)}, basicstyle=\ttfamily\scriptsize, caption={blablabla}, label={lst:1}]
\begin{lstlisting}
# npm install -g bower
# npm install -g grunt-cli
# npm install -g vows		(optional; runs unit tests)
$ npm install
\end{lstlisting}
\bigskip

\fi