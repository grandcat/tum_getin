\section{Design Alternatives}\label{sec:alt}
In the previous section, the overall architecture is presented without going into detail about the used hardware or how secure channels are implemented.
This section demonstrates competing drafts for the concrete design of central parts of our project. 
% By showing its advantages and drawbacks, it points out why different approaches are reasonable.
Due to dependent advantages and drawbacks, different approaches are reasonable for the same goal. These are discussed as follows.


\subsection{Hardware Setup}
A major part is a reliable communication link between an Android smartphone and a NFC reader.
To reach this goal, the NFC reader must support common NFC standards (e.g., ISO/IEC 14443) and provide stable transmission properties.
Therefore, we examine two different hardware setups to finally get a proper solution:
%
\begin{itemize}
	\item Raspberry Pi with Explore-NFC reader cape by NXP \footnote{\url{http://www.nxp.com/demoboard/PNEV512R.html}} or PN531 compliant cape.
	\item Smartphone with NFC capabilities and Android 4.4 or newer.
\end{itemize}
%
Using the Raspberry Pi as underlying platform and a NFC transceiver, a cheap and powerful setup is possible.
Providing a rich interface to interact with the environment, it allows an easy integration into existing door electronics.
For communication, either the vendors of these NFC capes provide SDKs or free libraries exist to abstract low-level functionality. 
However, the protocol layer is often incomplete and needs tweaks. This might influence the stability in the communication link.

On the other side, a smartphone with NFC transceiver and Android 4.4 (or higher versions) offers all required APIs in a simple way to exchange data between two Android smartphones over NFC.
Additionally, integrated capabilities like Wifi or Bluetooth allows enhanced applications. The downside of this approach are much higher costs, missing interfaces to interact directly with most door electronics and required user interaction for certain tasks (e.g., OS upgrade).


\subsection{Protocols for Authentication between Smartphone and NFC Reader}
A critical aspect in our project is the communication between a smartphone $ S $ and the NFC component $ T $.
It is important that it is safe so that no unauthorized party can easily gain access.
By placing the reader's antenna in the inner side of the door, it should be safe from physical threats from outside.
% In 

\subsubsection{Public-key Cryptography}

Protocol draft:$ R_S $: fresh random value by S

%%%%%%%%%%%%%%%%%%%%%%%%%%%%%%%%%%%%%%%%%%%%%%%%%%%%%%%%%%%%%%%%%%%%%%%%%%%
% Redefine the \mess due to problems with math support $ \some_function $ %
% See: http://tex.stackexchange.com/questions/164707/how-to-use-greek-    %
%      letters-in-pgf-umlsd-or-generally-terms-starting-with              %
%%%%%%%%%%%%%%%%%%%%%%%%%%%%%%%%%%%%%%%%%%%%%%%%%%%%%%%%%%%%%%%%%%%%%%%%%%%
\renewcommand{\mess}[4][0]{
  \stepcounter{seqlevel}
  \path
  (#2)+(0,-\theseqlevel*\unitfactor-0.7*\unitfactor) node (mess from) {};
  \addtocounter{seqlevel}{#1}
  \path
  (#4)+(0,-\theseqlevel*\unitfactor-0.7*\unitfactor) node (mess to) {};
  \draw[->,>=angle 60] (mess from) -- (mess to) node[midway, above]
  {#3};
}
%
\begin{sequencediagram}
	\newinst{S}{Smartphone $ S $}
	\newinst[9]{T}{NFC Reader $ T $}
	% \newinst[4]{rnc}{RNC}
	\mess{S}{1. $ {\{R_S, studentID, command\}}_{K_{T-pub}} $}{T}
	\postlevel
	\mess{T}{2. $ {\{R_S, R_T, T\}}_{K_{S-pub}} $}{S}
	\postlevel
	\mess{S}{3. $ {\{R_T, studentToken\}}_{K_{T-pub}} $}{T}
	%\mess{nodeb}{Synchronization Indication}{rnc}
	%\filldraw[fill=black!30] ($(RRC Connection Setup to)+(0,-.3)$) rectangle ($(Synchronization Indication from) +(0,.3)$)
	%node[midway] {L1 Synchronization};
\end{sequencediagram}
%
Explanation:
%
\begin{enumerate}
	% 1.
	\item $ S \rightarrow R $:
	\begin{itemize}
		\item $ S $ generates a random number $ R_S $ and sends it to $ T $ together with the studentID and further data, encrypted with the reader's public key $ K_{T-pub} $.
		\item The studentID can be a pseudonym associated with the official ID of this student. Like this, a MITM attack would not allow to directly gain information about the requesting person. 
	\end{itemize}	
	% 2.
	\item $ S \leftarrow R $:
	\begin{itemize}
		\item TODO
		\item Providing the identity $ T $ (i.e., hash of public key: $H(K_T)$), the requesting party can verify whether a MITM attack happened in 1. In that case, abort further communication.
	\end{itemize}	
	% 3.
	\item $ S \rightarrow R $:
	\begin{itemize}
		\item TODO
		\item $ studentToken $ : additional secret the attacker doesn't know beside the fresh $ R_T $ number.
	\end{itemize}
\end{enumerate}

Points to mention:
Key revocation

\subsubsection{Extended Randomized Hash Lock}
